%---------------------------------------------------
% Nombre: paquetes.tex  
% 
% Lista completa de paquetes empleados
%---------------------------------------------------
\usepackage{type1cm}					%paquete para redondear tama�os de fuente para evitar warnings
\usepackage[spanish, es-tabla]{babel}		%paquete para base de castellano para LaTex
\usepackage[T1]{fontenc}			%paquete de fuentes vectoriales
\usepackage{fix-cm}						%paquete para corregir problemas de fuentes
\usepackage[latin1]{inputenc} %paquete para usar acentos en ficheros .tex
\usepackage{graphicx}					%paquete para insertar im�genes
\usepackage{fancyhdr}					%paquete para modificar las cabeceras y pies de p�gina
\usepackage[colorlinks,
	    linkcolor=black,
            urlcolor=blue,
            citecolor=red]{hyperref} %paquete para modificar enlaces	
\usepackage{lettrine}					%paquete para incluir letras capitales
\usepackage{booktabs}					%paquete para realizaci�n de tablas profesionales
\usepackage{rotating}					%paquete para realizar rotaciones en tablas
\usepackage{color}						%paquete para emplear textos de color
\usepackage{algorithm}					%paquete para escribir c�digo
\usepackage{algpseudocode}
\usepackage{listings}

\definecolor{dkgreen}{rgb}{0,0.6,0}
\definecolor{gray}{rgb}{0.5,0.5,0.5}
\definecolor{mauve}{rgb}{0.58,0,0.82}

\lstset{frame=tb,
language=R,
numbers=left, numberstyle=\tiny, stepnumber=1, numbersep=-2pt,
aboveskip=3mm,
belowskip=3mm,
showstringspaces=false,
columns=flexible,
numbers=none,
keywordstyle=\color{blue},
numberstyle=\tiny\color{gray},
commentstyle=\color{dkgreen},
stringstyle=\color{mauve},
breaklines=true,
breakatwhitespace=true,
tabsize=3
}


\usepackage[normalem]{ulem}
\useunder{\uline}{\ul}{}
%\usepackage{anysize} 
%\input{Configuracion/spanishAlgorithmic}				% mi archivo de traducci�n
%---------------------------------------------------



